\documentclass[titlepage]{article}

\usepackage{preamble}

\usepackage{graphicx} % Required for inserting images

\title{
    \textbf{COMPARISON OF \\
    HEURISTIC ALGORITHMS \\
    FOR KP01} \\[0.5cm]
    \rule{12cm}{0.3mm} \\[0.5cm] 
    \small \scshape{A comparison of heuristic algorithms for solving the 0-1 Knapsack Problem} \\[0.4cm]
    \rule{12cm}{0.3mm}
    \vskip -0.3cm
}

\author{
    \small \scshape{Raunak Redkar} \\ 
    \small \scshape{Supervisors: Per-Olof Freerks and Felicia Dinnetz} \\
    \scriptsize \scshape{Kungsholmens Gymnasium}
}

\begin{document}

\onehalfspacing

\maketitle

\newpage
\tableofcontents
\newpage
\pagenumbering{arabic}

\section{Introduction}

\subsection{Background}

There are many situations in every day life, where a person wonders whether he is doing something efficiently. Unforunately, the human brain is not always capable of coming up with an optimal approach when the problem has a lot of factors at play. It is here when a system is used, and the true power of computing can be recognised. 

To solve such problems, a computer is provided all the information, from which it will produce an optimal answer. For the system to process all the information, certain instructions must be written into the system so that it knows how to handle the information. This set of instructions is called an algorithm. The system or computer uses algorithms to take in information (the input), and produce an answer - an output. The efficiency (in all aspects) of the algorithm is dependant on the instructions which make up it. There are many algorithms which have been designed to solve many problems. For problems in computer science, more than one algorithm is usually proposed and used. Optimization problems are where this is frequently the case. 

In the fields of computer science and mathematics, optimization problems are problems of finding the best solution, from a range of many feasible solutions. They are usually categorized into 2 categories: discrete optimizations and continuous optimizations, depending on whether the variables are discrete or continuous respectively. Combinatorial optimization problems are a subset of optimization problems that fall into the discrete. Combinatorial optimization involves searching for a maxima or minima for an objective function whose search space domain is a discrete but (usually large) space.

Typical combinatorial optimization problems are not limited to but include:
\begin{itemize}
    \item \textcolor{blue}{General Knapsack Problem} - Given a set of items, each with weight and profit value and a knapsack capacity, what is the best way to choose the items while respecting the knapsack capacity?
    \item \textcolor{blue}{Traveling Salesman Problem}- Given a list of cities, what is the shortest possible path that visits each city exactly once and returns to the origin?
    \item \textcolor{blue}{Set Cover} - Given a set of elements $\{1, 2, ..., n\}$ , what is the and a collection of $m$ sets whose union equals the universe, what is the smallest sub-collection of sets whose union is the universe?
\end{itemize}

The Knapsack Problem in particular has many variants which include the 0-1 knapsack problem, the bounded and unbounded knapsack problems, the multidimensional knapsack problem, the discounted knapsack problem, etc. The 0-1 Knapsack Problem is the simplest form of the knapsack problem and thus is also the one which has been worked on the most. 


\subsection{Aim}

\subsection{Research Question}

\section{Theory}




\end{document}
